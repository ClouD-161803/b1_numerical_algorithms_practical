\documentclass[hidelinks]{article}
%
%%%%%%%%%%%%%%%%%%%%%%%%%%%%%%%%%%%%%%%%%%%%%%%%%%%%%%%%%%%%%%%
% START CUSTOM INCLUDES & DEFINITIONS
%%%%%%%%%%%%%%%%%%%%%%%%%%%%%%%%%%%%%%%%%%%%%%%%%%%%%%%%%%%%%%%
%
\usepackage{amsmath}
\usepackage{parskip} %noident everywhere
\usepackage{hyperref} % Show hyperlinks - claudio
\hypersetup{
    colorlinks = true
    linkcolor = blue
    urlcolor = red
    }
% Block diagrams
\usepackage{tikz}
\usetikzlibrary{shapes.geometric, arrows, calc}
\tikzstyle{block} = [rectangle, draw,
    text centered, rounded corners, minimum height=3em, minimum width=6em]
\tikzstyle{sum} = [draw, circle, node distance=1cm]
\tikzstyle{input} = [coordinate]
\tikzstyle{output} = [coordinate]
\tikzstyle{arrow} = [draw, -latex']
%
%%%%%%%%%%%%%%%%%%%%%%%%%%%%%%%%%%%%%%%%%%%%%%%%%%%%%%%%%%%%%%%
% END CUSTOM INCLUDES & DEFINITIONS
%%%%%%%%%%%%%%%%%%%%%%%%%%%%%%%%%%%%%%%%%%%%%%%%%%%%%%%%%%%%%%%
%
\pdfobjcompresslevel=0
%
\title{\vspace{-4cm} Numerical Algorithms Report}
\author{\vspace{-2cm} Claudio Vestini}
\date{}
\begin{document}
\maketitle
%
\paragraph{Motivation}
This brief report will concern the B1 submarine coding practical, where
we were tasked with designing the controller to guide a Submarine through its cave mission by tracking a given reference to avoid collisions with the cave boundaries.
Preliminary steps taken before starting development:
%
\begin{enumerate}
    \item Create and activate a virtual environment with the given requirements (numpy, matplotlib and pandas packages).
    \item Fork project repository to my GitHub account.
    \item Set up a .env file to add local packages onto Python PATH.
    \item Make sure running files do not give any errors before branching off `main'.
\end{enumerate}
%
\paragraph{Mission Data}
The first task was to obtain mission data from the given .csv file. I achieved this through the following steps:
%
\begin{enumerate}
    \setcounter{enumi}{4}
    \item Create a new branch to modify the Mission class within.
    \item Implement a new classmethod to extract each column of the mission.csv file into a separate variable, then return the data as an instance of the Mission class.
    \item Test the new functionality by using the Trajectory class's plotting methods; then merge the branch back into `main', and delete the unused branch.
\end{enumerate}
%
\paragraph{Controller}
The design of the controller necessitated a thorough understanding of the Submarine and ClosedLoop classes, both of which were to be modified.
For full analysis, check Apppendix. I completed my implementation as follows:
%
\begin{enumerate}
    \setcounter{enumi}{7}
    \item Create a new branch to avoid pushing error prone code to `main'.
    \item Add a new method to the Submarine class to obtain the state space dynamics via matrices $A$, $B$, $C$ and $D$.
    \item Inside of a new module named control.py, create a new Controller class, initialised with the four matrices above. Create a subclass PDController, which inherits the dynamics, and possesses additional class variables $K_P$ and $K_D$. For this subclass, develop a class method to compute the next control action $u[t]$ given observation $y[t]$ and reference $r[t]$.
    \item Modify the ClosedLoop class to correctly call the controller within the simulate method.
    \item Test the controller by using the given demo.ipynb file. Merge and delete the branch.
\end{enumerate}
My decision to use a hierarchical class system proved effective as it kept the codebase modular. It also allows for future development of any other type of controller: I subsequently developed another subclass MPCController, which was better at tracking the reference but required more computational effort.
%
\end{document}